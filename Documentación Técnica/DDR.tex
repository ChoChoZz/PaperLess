\documentclass[11pt,letterpaper]{article}
\pagestyle{headings}
\usepackage[applemac]{inputenc}
\usepackage[T1]{fontenc}
\usepackage[swedish]{babel}
\usepackage{graphicx}
\usepackage{listings}
\usepackage[absolute]{textpos}

\setlength{\unitlength}{1 cm}
\setlength{\TPHorizModule}{30mm}
\setlength{\TPVertModule}{\TPHorizModule}
\textblockorigin{10mm}{10mm} % start everything near the top-left corner
\setlength{\parindent}{0pt}

\begin{document}
	\begin{titlepage}
		\begin{picture}(18,4)
			\put(-4,4){\includegraphics[width=4cm,height=3cm]{ipnLogo}}
			\put(10,4){\includegraphics[width=3cm,height=2.5cm]{escomLogo}}
		\end{picture}
		\begin{textblock}{4}(0.7,0.5)
			\begin{center}
				\textbf{\normalsize{INSTITUTO POLIT�CNICO NACIONAL}}\\[0.5cm]
				\textbf{\normalsize{ESCUELA SUPERIOR DE C�MPUTO}}\\[3cm]				
			\end{center}
		\end{textblock}
		
		\begin{textblock}{5}(0.7,2)
		\end{textblock}

	\end{titlepage}	
	
	\newpage
	\section{Introducci�n}
	El siguiente documento tiene la finalidad de dar a conocer los requerimientos funcionales y no funcionales del sistema que se implementar� en la Escuela Superior de C�mputo.\\
	Se especificar� la funcionalidad del sistema y se definir�n las condiciones de la implementaci�n.
	
	\subsection{�mbito del sitema}
	\subsection{Definciones, Acr�nimos y Abreviaturas}
	\subsection{Referencias}
	\subsection{Visi�n general del documento}
	
	\newpage

	\section{Descripci�n General}
	\subsection{Perspectiva del producto}
	\subsection{Funciones del producto}
	\subsection{Caracter�sticas de los usuarios}
	\subsection{Restricciones}
	
	\newpage
	\section{Requerimientos funcionales}
	\subsection{Definici�n de requerimientos funcionales}
	
	\begin{itemize}
		\item El administrador registra los usuarios del sistema y les da los permisos correspondientes para que accedan a secciones espec�ficas del sistema.
		\item El administrador gestiona la informaci�n de los departamentos.
		\item El Jefe de Departamento inicia sesi�n para acceder a la p�gina correspondiente de funciones del sistema.
		
	\end{itemize}
	
	\newpage
	\section{Requerimientos no funcionales}
	\subsection{Definici�n de requerimientos no funcionales}

	\newpage
	\section{Especificaci�n de requerimientos}	

	\newpage
	\section{Requerimientos del sitema}
	
	
	
\end {document}